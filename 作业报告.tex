\documentclass[UTF8]{ctexart}
\usepackage{listings}
\usepackage{xcolor}
\title{c++程序设计报告}
\author{计算机学院:刘健琛}
\date{\today}
\begin{document}
\maketitle
\tableofcontents
\section{作业题目}
用qt实现黑白棋游戏。
\section{开发环境}
\begin{itemize}
    \item qt creator 4.6.2
\end{itemize}
\begin{itemize}
    \item MINGW 5.3.0 32bit
\end{itemize}

\section{实现功能}
\begin{itemize}
    \item 8×8的棋盘与黑白棋子基本实现。
\end{itemize}
\begin{itemize}
    \item 黑白棋游戏规则与胜负判断函数的实现。
\end{itemize}
\begin{itemize}
    \item 两人单机对战,左右互搏的下棋方法实现。
\end{itemize}
\begin{itemize}
    \item 基于minimax剪枝算法的人工智能对弈方法实现。
\end{itemize}
\begin{itemize}
    \item qmessagebox的弹出窗口介绍。
\end{itemize}
\section{游戏规则}
\begin{itemize}
    \item 棋局开始时黑棋位于e4和d5,白棋位于d4和e5。

\end{itemize}
\begin{itemize}
    \item 黑方先行,双方交替下棋。
\end{itemize}
\begin{itemize}
    \item 一步合法的棋步包括:在一个空格新落下一个棋子,并且翻转对手一个或多个棋子。
\end{itemize}
\begin{itemize}
    \item 新落下的棋子与棋盘上已有的同色棋子间,对方被夹住的所有棋子都要翻转过来。可以是横着夹,竖着夹,或是斜着夹。夹住的位置上必须全部是对手的棋子,不能有空格。
\end{itemize}
\begin{itemize}
    \item 一步棋可以在数个方向上翻棋,任何被夹住的棋子都必须被翻转过来,棋手无权选择不去翻某个棋子。
\end{itemize}
\begin{itemize}
    \item 除非至少翻转了对手的一个棋子,否则就不能落子。如果一方没有合法棋步,也就是说不管他下到哪里,都不能至少翻转对手的一个棋子,那他这一轮只能弃权,而由他的对手继续落子直到他有合法棋步可下。
\end{itemize}
\begin{itemize}
    \item 如果一方至少有一步合法棋步可下,他就必须落子,不得弃权。
\end{itemize}
\begin{itemize}
    \item 棋局持续下去,直到棋盘填满或者双方都无合法棋步可下。
\end{itemize}
以上游戏规则同时被写在我的个人主页上,点击游戏中的开始按钮后会跳出窗口跳转到我的个人主页。


\section{基本程序代码的实现}
\subsection{画棋盘}
    \begin{lstlisting}[language={[ANSI]C},numbers=left,numberstyle=\tiny,%frame=shadowbox,  
       rulesepcolor=\color{red!20!green!20!blue!20},  
       keywordstyle=\color{blue!70!black},  
       commentstyle=\color{blue!90!},  
       basicstyle=\ttfamily]  
    
       void othello::paintEvent(QPaintEvent *event){
        this->resize(600,400);
        QPainter painter(this);
        painter.setRenderHint(QPainter::SmoothPixmapTransform,true);
        painter.drawPixmap(0,0,400,400,background);
        int c;
        if(turn=="b"){
            c=Whitechess;
        }
        else{
            c=Blackchess;
        }
        int potential[8][8];
        for(int i=0;i<8;i++){
            for(int j=0;j<8;j++){
                potential[i][j]=0;
            }
        }
        if(isGameOver(mainBoard)&&nom){
            QMessageBox msgBox;
            if(report(mainBoard,Whitechess)>report(mainBoard,Blackchess))
                msgBox.setText("white win.");
            else
                msgBox.setText("black win.");
            msgBox.exec();
            nom=false;
        }
       vector<posi> nodes =getMove(mainBoard,c);
       for(int i=0;i<nodes.size();i++){
           int x=nodes[i].x;
           int y=nodes[i].y;
           potential[x][y]=c;
       }
       for(int i=0;i<8;i++){
           for(int j=0;j<8;j++){
               if(mainBoard->get(i,j)==Whitechess){
                   painter.drawPixmap(0+50*i,0+50*j,50,50,white);
               }
               else if(mainBoard->get(i,j)==Blackchess){
                   painter.drawPixmap(0+50*i,0+50*j,50,50,black);
               }
               if(potential[i][j]==Blackchess){
                   painter.drawPixmap(0+50*i,0+50*j,50,50,hintblack);
               }
               if(potential[i][j]==Whitechess){
                   painter.drawPixmap(0+50*i,0+50*j,50,50,hintwhite);
               }
           }
       }
    }
    \end{lstlisting} 
通过一个有8×8的数组的棋盘类来存储当前棋盘上的棋子,用下载好的棋盘图片和自己画的棋子图片通过
pixmap来展示出来。这里可走棋子用圆圈来表示,已走的棋子用黑白实心棋子表现。
\subsection{鼠标点击实现下棋}
    \begin{lstlisting}[language={[ANSI]C},numbers=left,numberstyle=\tiny,%frame=shadowbox,  
    rulesepcolor=\color{red!20!green!20!blue!20},  
    keywordstyle=\color{blue!70!black},  
    commentstyle=\color{blue!90!},  
    basicstyle=\ttfamily]  
    mousex=e->x()/50;
        mousey=e->y()/50;
        if(turn.compare("a")==0&&isGameOver(mainBoard)!=true&&!gameOver){
            if(makeMove(mainBoard,playera,mousex,mousey)==true){
                Last.x=mousex;
                Last.y=mousey;
                if(getMove(mainBoard,playerb).size()!=0){
                    turn="b";
                }
            }
        \end{lstlisting} 
\subsection{可走棋子判定函数}
\begin{lstlisting}[language={[ANSI]C},numbers=left,numberstyle=\tiny,%frame=shadowbox,  
    rulesepcolor=\color{red!20!green!20!blue!20},  
    keywordstyle=\color{blue!70!black},  
    commentstyle=\color{blue!90!},  
    basicstyle=\ttfamily]  
vector<posi> othello::isValid(Board *board, int tile, int col, int row){
    vector<posi> tilesToFlip;
        if (isOnBoard(col, row) == false || board->get(col, row) != 0) {
            return tilesToFlip;
        }
        int othertile;
        board->set(col, row, tile);

        if (tile == Blackchess) {
            othertile = Whitechess;
        }
        else {
            othertile = Blackchess;
        }
        for (int i = 0; i < 8; i++) {
            int x = col;
            int y = row;
            int xdirection = Direction[i][0];
            int ydirection = Direction[i][1];
            x += xdirection;
            y += ydirection;
            if (isOnBoard(x, y) && board->get(x, y) == othertile) {
                x += xdirection;
                y += ydirection;
                if (isOnBoard(x, y) == false) {
                    continue;
                }
                while (board->get(x, y) == othertile) {
                    x += xdirection;
                    y += ydirection;
                    if (isOnBoard(x, y) == false) {
                        break;
                    }
                }
                if (isOnBoard(x, y) == false) {
                    continue;
                }
                if (board->get(x, y) == tile) {
                    while (true) {
                        x -= xdirection;
                        y -= ydirection;
                        if (x == col && y == row) {
                            break;
                        }
                        tilesToFlip.push_back(posi(x, y));
                    }
                }
            }
        }
        board->set(col, row, 0);
        return tilesToFlip;
}
\end{lstlisting} 
通过规则已知这里只有能够将对方棋子反转的下法才可以落子,那么我们就通过向八个方向遍历
看能否用两个同色棋子将对方连续棋子夹住。
\subsection{minimax方法}
\begin{lstlisting}[language={[ANSI]C},numbers=left,numberstyle=\tiny,%frame=shadowbox,  
    rulesepcolor=\color{red!20!green!20!blue!20},  
    keywordstyle=\color{blue!70!black},  
    commentstyle=\color{blue!90!},  
    basicstyle=\ttfamily]  
posi othello::max(Board *mb, int depth, int alpha, int beta, int tile,int hard){
    int best = -10000;
    Board *nm=new Board();
    posi move ;
    for(int i=0;i<8;i++){
        for(int j=0;j<8;j++){
            nm->set(i,j,mb->get(i,j));
        }
    }
    vector<posi> gm=getMove(mb,tile);
    if(depth==0){
        for(int i=0;i<getMove(mb,tile).size();i++){
            nm->set(gm[i].x,gm[i].y,tile);
            if(evaluate(nm,hard)>best){
                best=evaluate(nm,hard);move=gm[i];
            }
            nm->set(gm[i].x,gm[i].y,0);
        }
        return move;
    }
    if(gm.size()==0){
        return move;
    }
    for (int i = 0; i < getMove(mb,tile).size(); i++) {
        alpha = best>alpha?best:alpha;
        if(alpha >= beta){
            break;
        }
        nm->set(gm[i].x,gm[i].y,tile);
        posi next;
        next=min(nm, depth - 1,alpha, beta, -tile, hard);
        nm->set(next.x,next.y,-tile);
        int value = evaluate(nm,hard);
        if (value < best) {
          best = value;
          move = gm[i];
        }
        nm->set(next.x,next.y,0);
        nm->set(gm[i].x, gm[i].y,0);
      }
      return move;
}
\end{lstlisting}


minimax方法有min函数和max函数,对当前局面进行评估,这里截取max函数,该函数的逻辑是要让落子之后,对方的min取值
只能最大。而min函数同理,通过两边对抗搜索,来寻找对各自最好的情况。搜索多层就可以得到期望的更优选择。

这里的评估函数用一个8×8的矩阵加权算得,在这里黑棋的赋值是1,白棋的赋值是-1(由此一来黑棋用max求优,白棋用min)。

而难度可以由搜索层数和评估矩阵的合理性来决定。
\section{游戏说明}


在开始玩之前在qbuttongroup中选择是ai模式还是人人模式,可以在lineedit中输入难度,
然后点击start按钮,跳出窗口介绍规则。此时start按钮变成reset按钮,再按它就会恢复棋盘。

这之后就可以开始下了,如果是人人模式,可以找一个同学一起下。你在棋盘上选择一个被黑/白圆圈圈出来的棋盘
格子点击就可以落子了(如果没有圈圈代表你这一轮没得走),然后可以请你的同学来接着下另一颜色的棋子。
双方轮流下直到游戏结束。

如果是人机模式,你默认了先下的黑棋。在你点击完黑棋之后,电脑会自动下出白棋的落子,
两方互相下棋,直到游戏结束。
\section{收获}

通过此次作业,了解了qt中的各个控件和使用方法,比如lineedit,qmessagebox,qlabel,qpixmap等。


同时,对minimax对抗搜索方法有了一定的了解,在此前的作业中了解了蒙特卡洛树搜索,这次作业
对ai的对抗搜索有了更深的理解。
\end{document}